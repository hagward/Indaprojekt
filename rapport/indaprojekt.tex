\documentclass[11pt,a4paper]{article}
\usepackage[utf8]{inputenc}
\usepackage[swedish]{babel}
\usepackage{fancyhdr}

\author{Anders Hagward \and Fredrik Hillnertz}
\title{Slutprojekt}

\pagestyle{fancy}
\fancyhead{}
\fancyfoot{}
\lhead{Slutprojekt i inda}
\cfoot{\thepage}

\begin{document}

\thispagestyle{empty}
\begin{center}
\huge Slutprojekt i Introduktion till Datalogi

\normalsize Anders Hagward och Fredrik Hillnertz
\end{center}
\newpage

\tableofcontents
\newpage

\section{Projektplan}

\subsection{Programbeskrivning}
Programmet skall underhålla två användare genom att låta dessa kriga mot varandra med var sin virtuell stridsvagn.

\subsection{Användarbeskrivning}
Vi förutsätter att användarna har ``lagom'' datorvana, det vill säga att de vet hur man startar en dator, kör ett program och interagerar med mus och tangentbord. En ungefärlig åldersskala är 8-60 år.

\subsection{Användarscenarier}
Spelet kräver två användare, nedan kallade * och **.

\subsubsection{Scenario I}
\begin{enumerate}
	\item * kör programmet och möts av en meny med valen \emph{New Game} och \emph{Quit}.
	\item * väljer det första alternativet och ett nytt spel startas.
	\item En översiktsvy över en tvådimensionell karta presenteras. På kartan är två stridsvagnar och ett antal hinder utplacerade.
	\item * styr den högra stridsvagnen med piltangenterna och avfyrar skott med högra CTRL-knappen. ** styr den vänstra med tangenterna W, A, S och D och skjuter med vänstra CTRL-knappen.
	\item * lyckas förstöra **:s stridsvagn tre gånger genom att skjuta på den och vinner spelet.
	\item en meny med valen \emph{Play Again} och \emph{Quit} visas.
	\item * väljer det senare alternativet och spelet avslutas.
\end{enumerate}

\subsubsection{Scenario II}
TODO

\subsection{Testplan}
Vi skall testa ett flygplan.

\subsection{Programdesign}
Programmet skall vara designat enligt senaste mode.

\subsection{Tekniska frågor}
Vänligen kontakta Fredrik och klaga.

\subsection{Arbetsplan}
Arbetet kommer ske enligt en plan.

\end{document}