\documentclass[11pt,a4paper]{article}
\usepackage[utf8]{inputenc}
\usepackage[swedish]{babel}
\usepackage{fancyhdr}

\author{Anders Hagward \and Fredrik Hillnertz}
\title{Slutprojekt}

\pagestyle{fancy}
\fancyhead{}
\fancyfoot{}
\lhead{Slutprojekt i inda}
\cfoot{\thepage}

\begin{document}

\thispagestyle{empty}
\begin{center}
\huge Slutprojekt i Introduktion till Datalogi

\normalsize Anders Hagward och Fredrik Hillnertz
\end{center}
\newpage

\tableofcontents
\newpage

\section{Projektplan}

\subsection{Programbeskrivning}
Programmet skall underhålla och stjäla tid från användaren.

\subsection{Användarbeskrivning}
Användaren förutsätts ha ``lagom'' datorvana, det vill säga vetskap om hur man startar en dator, kör ett program och interagerar med mus och tangentbord. En ungefärlig åldersskala är 6-60 år.

\subsection{Användarscenarier}
Nedan följer två vanliga scenarier.

\subsubsection{Scenario I}
\begin{enumerate}
	\item Användaren startar programmet och möts av en vy innehållandes ett block längst ned som går att styra, en hord av block längst upp som ej går att styra samt en studsande boll.
	\item Användaren styr bottenblocket genom att röra på datormusen så att bollen studsar på det.
	\item Blocken längst upp försvinner successivt i och med att de får kontakt med bollen.
	\item Alla 'toppblock' har slutligen försvunnit och användaren vinner spelet.
	\item En meny med valen \emph{Play again} och \emph{Quit} visas och användaren väljer det sistnämnda.
\end{enumerate}

\subsubsection{Scenario II}
\begin{enumerate}
	\item Användaren startar programmet och möts av samma vy som i föregående scenario.
	\item Användaren styr bottenblocket genom att röra på datormusen men lyckas inte få bollen att studsa på det.
	\item Användaren missar bollen tre gånger och förlorar därmed spelet.
	\item En meny med valen \emph{Play again} och \emph{Quit} visas. Användaren väljer att avsluta.
\end{enumerate}

\subsection{Testplan}
Vi skall testa ett flygplan.

\subsection{Programdesign}
Programmet skall vara designat enligt senaste mode.

\subsection{Tekniska frågor}
Vänligen kontakta Fredrik och klaga.

\subsection{Arbetsplan}
Arbetet kommer ske enligt en plan.

\end{document}